\section{Класс $\poly$}

\begin{definition}[Класс P]
	$\poly =\cup_{c=1}^{\infty} \Dtime\left(n^c\right)$
\end{definition}

Очевидно, что не все языки лежат в данном классе.
Однако большинство привычных вам задач в данном классе лежит.

\begin{problem}
Приведите пример языка $L$ такого, что он не принадлежит $\poly$,
а $L^*$ принадлежит $\poly$.
\end{problem}
\begin{solution}
	В $\poly$, по очевидным соображениям, не лежат неразрешимые языки,
	но лежит язык всех слов.
	Тогда в качестве $L$ можно взять язык $\halt \cup \{0, 1\}$.
\end{solution}

\begin{problem}[Примеры P-задач]
Приведем обоснования принадлежности некоторых языков к P:
\begin{enumerate}
	\item $\mathrm{GCD} = \{ (a, b, d) \mid
		      \text{число $d$ является наибольшим общим делителем чисел $a$ и $b$} \}$;
	\item $\mathrm{PATH}= \{ (G, s, t) \mid
		      \text{в графе $G$ есть путь из $s$ в $t$} \}$;
	\item $\mathrm{CONNECTED} = \{G \mid
		      \text{ $G$ --- связный граф } \}$;
	\item $\mathrm{EULERCYCLE} =\{G \mid
		      \text{в графе $G$ есть эйлеров цикл} \}$.
\end{enumerate}
\end{problem}
\begin{solution}
	\begin{enumerate}
		\item Действительно, наибольший общий делитель можно найти
		      алгоритмом Евклида и затем сравнить результат с $d$.
		\item Наличие пути в графе проверяется быстро, например обходом в ширину.
		\item Например, можно проверить граф на связность,
		      проверив наличие пути из некоторой фиксированной вершины во все остальные,
		      или использовать обход графа единовременно.
		\item По критерию существования эйлерова цикла достаточно проверить связность,
		      а также посчитать степени всех вершин и проверить, что все они чётные.
	\end{enumerate}
\end{solution}

Исследуемый класс замкнут относительно всех теоретико-языковых операций.

\begin{theorem}[Замкнутость P]
	Класс P замкнут относительно:
	\begin{enumerate}
		\item объединения;
		\item пересечения;
		\item дополнения;
		\item конкатенации;
		\item итерации;
	\end{enumerate}
\end{theorem}
\begin{proof}
	Первые три утверждения тривиальны: можем запустить МТ для наших языков,
	подождать их ответов и получить из них финальный результат.

	Справедливость замкнутости относительно конкатенации следует из следующего:
	всего у нас есть $O(n)$ различных разбиений слова
	на две части (левую и правую).
	Каждую из них мы за полином можем проверить на принадлежность
	соответствующему языку, таким образом получаем полиномиальность конкатенации.

	Теперь разберемся с итерацией.
	Идея в том, чтобы выяснять принадлежность языку $L^*$
	префиксом слова (по очереди), тогда, в итоге, мы выясним,
	принадлежит ли $L^*$ само слово $w$. Обозначим $p(k)$ --- префикс длины $k$.

	\begin{itemize}
		\item Вначале имеем $p(0)=\varepsilon \in L^*$ по определению звезды Клини.
		\item Пусть мы выяснили принадлежность $L^*$ всех
		      $p(i)$ для $i \leqslant k$. Пусть $p(k+1)=s$.
		      Рассмотрим все его разбиения вида $s=x y, y \neq \varepsilon$.
		      Для каждого такого разбиения мы можем с помощью известного алгоритма
		      выяснить принадлежность $y$ к $L$, а принадлежность $x$ к $L^*$
		      нам уже известна. Всего таких шагов $O(|w|)$, каждый занимает
		      $O(|w|)\cdot O(p(|w|))$
		\item Таким образом мы дойдем до префикса,
		      являющегося всем словом $w$, и получим ответ на интересующий нас вопрос
		      о принадлежности $w$ языку $L^*$.
	\end{itemize}
\end{proof}

По описанию машины нельзя понять, работает ли она полиномиальное время.
Более того, язык описаний таких машин нельзя даже перечислить.
Но интересно, что можно перечислить список полиномиальных машин такой,
что все машины в списке совместно разрешают все языки из $\poly$ и только их.

\begin{problem}
\begin{enumerate}
	\item Докажите, что язык $L_p = \{\langle M \rangle \mid
		      \text{$M$ работает полиномиальное время} \}$
	      не принадлежит классу $\enum$.
	\item Докажите, что существует перечислимый язык $L_l$, такой, что
	      для каждого языка $L \in \poly$ в $L_l$
	      есть описание машины, распознающей $L$,
	      и любая машина, описание которой принадлежит $L_l$
	      работает полиномиальное время.
\end{enumerate}
\end{problem}
\begin{solution}
	Покажем, что $L_p \not \in \enum$ построив сводимость $\nothalt \leq L_p$.
	Пусть сводящая функция $f$ отображает пару $(M, x)$ в машину $A_x$.
	$A_x$ работает на слове $y$ так: она запускает $M$ на слове $x$
	и принимает слово $y$ если та отработала $|y|$ тактов.
	Если $M$ зависает на $x$, то $A_x$ принимает все слова и работает
	полиномиально. Если $M$ останавливается на $x$, то $A_x$ останавливается
	не на всех словах.

	Будем строить перечисляющую машину для $L_l$ следующим образом:
	пусть $A_i$ --- перечисление вообще всех машин Тьюринга.
	Перечисление языка $L_l$ будем делать по шагам: на шаге $n$
	выдаем первые $n$ машин из $A_i$ с ограничением времени работы
	до $i x^i$ тактов. Все перечисленные машины работают полиномиально,
	при этом для любого языка $L \in \poly$ найдется машина в $A_i$ и подходящее
	ограничение на количество тактов, которые задают машину из $L_l$,
	распознающую $L$.
\end{solution}