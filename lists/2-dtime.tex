\section{Классы $\Dtime$}

Дадим определение класса языков, распознаваемых за некоторое время на
$k$-ленточной ДМТ.
\begin{definition}[$\Dtime$]
	Язык $L$ лежит в классе $\Dtime(T(n))$ тогда и только тогда, когда
	существует ДМТ, которая распознает его за $O(T(n))$.
\end{definition}

Покажем, что получившиеся классы строго вложены друг в друга
на примере следующей задачи.

\begin{problem}[$\Dtime(n) \subsetneq \Dtime(n^{1.5})$]
Докажите, что $\Dtime(n) \subsetneq \Dtime(n^{1.5})$.
\end{problem}

\begin{solution}
	Рассмотрим машину Тьюринга $D$:
	На входе $x$ она запускает универсальную МТ $U(x, x)$
	на $|x|^{1.4}$ шагов.
	Если $U$ принимает слово за это время,
	то $D$ его отвергает и наоборот.
	Если $U$ не дало ответа за это время,
	то $D$ отвергает слово.

	Пусть $L$ --- язык слов, которые принимает машина $D$.
	По определению, $D$ распознает язык $L$ за $|x|^{1.4}$,
	а значит он лежит в $\Dtime(n^{1.5})$.
	Докажем от противного, что он не лежит в $\Dtime(n)$.

	Предположим, что существует машина $M$,
	такая что она завершает работу на любом слове $x$
	за $c_1|x|$ шагов и выдает ответ $D(x)$.
	Мы можем вычислить
	$U(M, x)$ за $c_2(c_1|x|)\log(c_1|x|) = a|x|\log|x|$.
	Выберем такое число $n_0:\ \forall n \geq n_0:\ n^{1.4} > an\log n$.
	Теперь выберем $y$ --- описание машины $M$,
	такое что оно не короче $n_0$.

	Тогда $D$ выдаст на $y$ ответ обратный $U(y,y)=M(y)$.
	Но по предположению $D(y) = M(y)$. Пришли к противоречию.
\end{solution}

Данная задача имеет обобщение, которое называется
    теоремой о временной иерархии. Для этого нам потребуется
определить класс конструируемых по времени функций,
к которым она применима.
Все <<обычные>> функции ($n^c, 2^n, n \log n$ и т.д.)
конструируемы по времени.

\begin{definition}[Функция конструируемая по времени]
	Функция $T:\ \Nat \ra \Nat$ называется конструируемой по времени,
	если существует алгоритм,
	который за время $O(T(n))$ по $1^n$ получает $T(n)$ в бинарной записи.
\end{definition}

\begin{theorem}[Об иерархии по времени]
	Пусть $f: \mathbb{N} \rightarrow \mathbb{N}$ и
	$g: \mathbb{N} \rightarrow \mathbb{N}$ ---
	строго монотонные конструируемые по времени функции,
	такие что $f(n) \log f(n)=$ $o(g(n))$.
	Тогда $\Dtime(f(n)) \subsetneq \Dtime(g(n))$.
\end{theorem}

\begin{proof}
	Смотри в \cite{musatych}.
\end{proof}

Однако у классов $\Dtime$ есть существенный изъян ---
неустойчивость к выбору модели вычислений.

\begin{problem}[Об языке палиндромов]
\begin{enumerate}
	\item Покажите, что язык палиндромов над алфавитом $\{0, 1\}$
	      распознается за $T(n)=O(n)$ на многоленточной МТ.
	\item Докажите, что любая одноленточная МТ,
	      зык палиндромов над алфавитом $\{0, 1\}$,
	      работает за время $T(n)=$ $\Omega\left(n^2\right)$.
\end{enumerate}
\end{problem}
\begin{solution}
	Первый пункт тривиален: у МТ есть две ленты,
	она переписывает слово на вторую ленту,
	после чего идет по одной слева направо,
	а по другой справа налево и сравнивает символы.

	Перейдем ко второму пункт.
	Рассмотрим слова вида $x 0^{2n} y^R$,
	где $|x| = |y| = n$.
	Пусть протокол в точке $i$ --- это последовательность состояний, в которых машина
	перешла из $i$-ой в $(i + 1)$-ю ячейку или наоборот.
	Тогда при $x \neq z$ и для любого $i: n < i < 3n$,
	протоколы работы машины на словах $x0^{2n}x^R$ и $z0^{2n}z^R$ различны,
	иначе бы она принимала и слово $x0^{2n}z^R$.

	Далее, существует некоторое $i$, т.ч.
	протокол в точке $i$ --- самый короткий (среди всех $2n + 1$ протоколов)
	хотя бы для $2^n/(2n + 1)$ из слов вида $x 0^{2n} x^R$ (по принципу Дирихле).
	Для всех этих слов протоколы в $i$ различны,
	так что есть слово с минимальной длиной протокола $\Omega(n)$.
	Наконец, каждому элементу такого протокола соответствует такт работы машины.
	Значит, тактов хотя бы на одном слове совершается $\Omega(n^2)$,
	так как все его $2n+1$ протокола имеют $\Omega(n)$ слов.
\end{solution}

Однако, как мы знаем из тезиса Чёрча-Тьюринга, разница во времени распознавания
языка с использованием разных моделей вычислений не более чем полиномиальна.

\begin{problem}[Об избавлении от лент]
Любой язык, который можно распознать за время $T(n)$
на многоленточной машине Тьюринга,
можно распознать за время $O(T(n)^2)$ на одноленточной машине.
\end{problem}
\begin{solution}
	Будем моделировать машину с $k$ лентами при помощи одноленточной.
	На ленте новой машины будем хранить содержимое всех лент исходной,
	а также положения всех указателей.
	Поскольку исходная машина работает не дольше $T(n)$,
	то на каждой из лент непустыми могут быть не больше $T(n)$ ячеек.
	Тогда заранее на одноленточной машине разметим пространство
	под каждую ленту исходной, поставив разделители.
	Таким образом, новая машина займёт не больше $kT(n)$ ячеек,
	т.е. $O(T(n))$: число лент может быть сколь угодно большим, но не зависит от длины входа.
	Для моделирования одного шага исходной машины нужно изменить содержимое всех лент
	и передвинуть указатели. Моделирование одного шага потребует $O(T(n))$ шагов,
	таким образом моделирование всей работы потребует
	$O(T(n)) \cdot O(T(n)) = O(T(n)^2)$ шагов.
\end{solution}

Это приводит нас к следующей идее: интересный для изучения класс задач,
свойства которого не зависят от выбора модели вычислений --- это
класс задач решаемых за полиномиальное от длины входа времени на ДМТ.